\documentclass[t, 8pt]{beamer}
\usetheme{iclpt}
\usepackage{url}

\author{Martyn Jarvis}
\title[Electron Charge Asymmetry]{Electron Charge Asymmetry in $pp \to W+X \to
lv + X$ production @ $E_{cm} = 7$TeV}

\begin{document}
\begin{frame}
  \titlepage
\end{frame}

\begin{frame}{Outline}
  \tableofcontents
\end{frame}

\section{Introduction}

\begin{frame}{Motivation}
  \begin{columns}[c]
  \column{.5\textwidth}
  \begin{itemize}
    \item In pp collisions, more $W^+$ are expected than $W^-$ due to the excess of u valence quarks wrt d quarks.
    \item An asymmetry measurement as a function of boson rapidity can be used to constrain PDFs.
    \item Boson rapidity is not directly accessible
    \item Direct accessible measurement is the lepton charge asymmetry
  \end{itemize}
  \column{.5\textwidth}
    \includegraphics[width=\textwidth]{fig/rapidity}
  \end{columns}

    \begin{equation}
    A_{th}(\eta) = \frac{\frac{d\sigma}{d\eta}(W^{+}\to e^{+}\nu_{e}) - \frac{d\sigma}{d\eta}(W^{-}\to e^{-}\nu_{e})}
                        {\frac{d\sigma}{d\eta}(W^{+}\to e^{+}\nu_{e}) + \frac{d\sigma}{d\eta}(W^{-}\to e^{-}\nu_{e})}
    \end{equation}
  \begin{itemize}
    \item This asymmetry is given by the combination of the W production asymmetry and the well understood parity violation asymmetry in the W decay.
  \end{itemize}
\end{frame}

\begin{frame}{Motivation}
  \begin{columns}[c]
  \column{.5\textwidth}
  \begin{itemize}
    \item Lepton charge asymmetry and W charge asymmetry have been studied at Tevatron
    \item Several authors have reported tension between the lepton charge asymmetry and PDF global fits
    \item Current predictions for the asymmetry at the LHC do not agree
  \end{itemize}
  \column{.5\textwidth}
    \includegraphics[width=\textwidth]{fig/asym_theory}
  \end{columns}
\end{frame}



\section{History}

\begin{frame}{History}
  \begin{columns}[c]
  \column{.5\textwidth}
  \begin{itemize}
    \item Previous CMS measurement with the full 2010 dataset of $36 pb^{-1}$ in both electron and muon channel (right)
    \item Lepton charge asymmetry and W charge asymmetry also been studied at ATLAS and LHCb
    \item Muon channel measurement using $234pb^{-1}$ of 2011 dataset
    \vspace{1cm}
    \item This new measurement has been performed with $840pb^{-1}$ of data from the 2011 dataset.
  \end{itemize}
  \column{.5\textwidth}
    \includegraphics[width=\textwidth]{fig/2011_asym_combined}
  \end{columns}
\end{frame}

%\section{LHC Performance}


%\begin{frame}{LHC Performance in 2011}
  %\begin{columns}[c]
  %\column{.5\textwidth}
  %\begin{itemize}
  %\end{itemize}
  %\column{.5\textwidth}
    %%\includegraphics[width=\textwidth]{fig/overtrain_old_same_script}
  %\end{columns}
%\end{frame}

%\section{CMS Detector}

%\begin{frame}{CMS Detector}
    %\includegraphics[width=\textwidth]{fig/overtrain_old_same_script}
%\end{frame}

\section{Analysis}

\begin{frame}{Analysis Overview}
  \begin{columns}[c]
  \column{.5\textwidth}
  \begin{itemize}
    \item $W \to e\nu$ characterised by 
    \begin{itemize}
      \item High Pt lepton 
      \item Missing transverse energy (MET) due to neutrino
    \end{itemize}
    \item Background contributions,
    \begin{itemize}
      \item EWK background ( $W \to \tau\nu$, Drell-Yan)
      \item $t\bar{t}$ background
      \item QCD background (multi-jet, photon+jet)
    \end{itemize}
    %\item Lepton pT> 35 GeV (limit set by single electron trigger) 
    \item The number of the electron / positron is extracted from a extended binned likelihood fit of the MET distribution
    \item 11 bins in $|\eta|$
    \begin{itemize}
      %\item  $[0-0.2], [0.2-0.4], [0.4-0.6], [0.6-0.8], [0.8-1.0], [1.0-1.2], [1.2-1.4], [1.6-1.8], [1.8-2.0], [2.0-2.2], [2.2-2.4]$
      \item $0.0 < |\eta| < 2.4$ in bin widths of 0.2
      \item $[1.4-1.6]$ bin excluded because of the transition region between Ecal Barrel and Ecal Endcap
    \end{itemize}
  \end{itemize}
  \column{.5\textwidth}
    \includegraphics[width=\textwidth]{fig/wenu_event}
  \end{columns}
\end{frame}

\begin{frame}{Event Selection}
  \begin{columns}[c]
  \column{.5\textwidth}
  \begin{itemize}
    \item Electron Selection:
    \begin{itemize}
      \item $P_T > 35 GeV$
      \item Constrained by trigger
    \end{itemize}
    \item Electron Identification
    \begin{itemize}
      \item Track cluster matching
      \item Shower shape and H/E
      \item Track, ECAL and HCAL isolation
      \item Conversion rejection
    \end{itemize}
    \item Z veto
    \begin{itemize}
      \item 2nd lepton with $P_T > 15$ GeV
    \end{itemize}
    \item Require that all three methods of charge assignment agree to reduce
    misassignment (Gaussian Sum Filter, Kalman Filter, Relative phi position of
    cluster center and first tracker hit)
  \end{itemize}
  \column{.5\textwidth}
    %table of cuts
    \tiny{
  \begin{center}

    %table with 
    \begin{table}[htbp]
    \begin{tabular}{|l|r|}
    \hline
    & $P_{T}>35$ GeV \\ \hline
    $W\to e\nu$  & 76.2$\%$\\\hline
    QCD Background       & 16.0$\%$\\\hline
    EWK Total Background & 7.8$\%$ \\
    EWK DYtautau         & 0.2$\%$  \\
    EWK DYee             & 6.4$\%$  \\
    EWK Wtaunu           & 0.8$\%$ \\
    EWK ttbar            & 0.4$\%$ \\\hline
    \end{tabular}
    \caption{\label{tab:composition} Composition of selected events.}%from simulation.}
    \end{table}
  \end{center}
  }

    %\includegraphics[width=\textwidth]{fig/overtrain_old_same_script}
  \end{columns}
\end{frame}

\begin{frame}{Selected Events}
  %\begin{columns}[c]
  %\column{.5\textwidth}
  \begin{itemize}
    \item Apply electron selection, then extended binned maximum likelihood fit to the MET distribution for electrons and positrons separately
    \item Two template shapes:
    \begin{itemize}
      \item Signal + EWK backgrounds : MC + correction from $Z \to ee$ recoil in data 
      \item QCD shape is determined using a signal-free control sample obtained by inverting a subset of the electron ID criteria
    \end{itemize}
  \end{itemize}
  %\column{.5\textwidth}
    % FIT here
    \begin{center}
      \includegraphics[width=0.8\textwidth]{fig/fig1a}
    \end{center}
  %\end{columns}
\end{frame}

\section{Systematics}

%\begin{frame}{Systematics and Corrections}
%  \begin{itemize}
%    \item Four main sources of systematic uncertainty in both channels:
%    \begin{itemize}
%      \item Relative efficiencies between $e^{+}$ and $e^{-}$
%      \item Lepton momentum (energy) scale and resolution
%    \end{itemize}
%  \end{itemize}
%\end{frame}

%\begin{frame}{Signal Extraction Method}
%  \begin{itemize}
%    \item Systematic uncertainties evaluated by varying the shapes used in the fits
%    \item Signal Shape
%    \begin{itemize}
%      \item Uncertainty in the recoil correction
%      \item Theoretical input to the recoil due to PDF uncertainty
%    \end{itemize}
%    \item QCD Shape
%    \begin{itemize}
%      \item Anti-selection was varied to study the effect on the fit. 
%    \end{itemize}
%    \item EWK Shape
%    \begin{itemize}
%      \item The relative normalizaton of the $Z \to ee$, $W \to \tau\nu$,
%      $t\bar{t}$, is modified by $20\%$
%    \end{itemize}
% \end{itemize}
%\end{frame}

\begin{frame}{Systematics and Corrections}
  %\begin{columns}[c]
  %\column{.5\textwidth}
  \begin{itemize}
    \item To compare with the theoretical value, the measured lepton asymmetry has to be corrected for experimental effects:
    \begin{itemize}
      \item Charge mis-id $(\omega)$: 
      \begin{itemize}
        \item Introduces a dilution factor to the real asymmetry
        \item Evaluated from the ratio same-sign / opposite-sign Z yield
        %\item  $R_{ij}$=fraction of the same sign events with one electron in the η bin i and the other in the bin j
        %\item From a simultaneous fit to the 66 values of $R_{ij}$, the value of $\omega$ is evaluated for each $\eta$ bin
        \item Statistical error propogated to asymmetry as systematic error
      \end{itemize}
      \item Relative detection efficiency $(R = \epsilon^+/\epsilon^-)$ between electrons and positrons: 
      \begin{itemize}
        \item the detection efficiency is different for electrons and positrons and produces a bias in the measured asymmetry
        \item Measured with tag and probe.
        \item Systematic error (energy scale and signal shape) cancel out in the ratio R, only the statistical error is propagated to R
      \end{itemize}
      \begin{center}
      \begin{equation}
        \mathcal{A}_R=
          \frac{1}{1-2\omega}\frac{ \mathcal{A}_M\left(R+1\right) - \left(R-1\right)}{\left(R+1\right)-\mathcal{A}_M \left(R-1\right)}
          \simeq 
          \frac{1}{1-2\omega}\left(\mathcal{A}_M -\frac{\left(R-1\right)\left(1-\mathcal{A}_M^2\right)}{2}\right)
      \end{equation}
      \end{center}
    \end{itemize}
  \end{itemize}
\end{frame}

\begin{frame}{Systematics and Corrections}
  \begin{itemize}
    \item Signal extraction method
    \begin{itemize}
      \item Systematic uncertainties evaluated by varying the shapes used in the fits
    \end{itemize}
    \item Energy Scale and Resolution
    \begin{itemize}
      \item The electron energy resolution and scale can introduce a bias on the
      asymmetry, due to the effect of resolution for leptons with a transverse
      momentum close to $P_T$ cut.
      \item The correction factor has been evaluated by comparing the asymmetry at gen level and the asymmetry after simulating the particle-matter interaction
      \item An additional data/mc correction is applied to account for differences between data and simulation
      \begin{equation}
      P_T^{sim} = P_T^{gen} \otimes Res^{MC} \otimes Res^{Data/MC}
      \end{equation}
      \item where:
      \begin{itemize} 
        \item $P_T^{gen}$ is from resbos with CT10
        \item $Res^{MC}$ is from a $(W\to e\nu)$ MC sample
        \item $Res^{Data/MC}$ is from a residual data/MC correction 
      \end{itemize}
    \end{itemize}
  \end{itemize}
\end{frame}

\begin{frame}{Summary of Systematic Uncertainties}
  \begin{columns}[c]
  \column{.6\textwidth}
  \tiny{
  \begin{center}
    \begin{table}
    \begin{tabular}{l|c|cccc}
      \hline\hline
      &Stat  &Signal & Energy & Charge &  Efficiency \\
     & Error &Yield & Scale and Res. & MisId. & Ratio \\ \hline
      $0.0<|\eta|<0.2$& 3& 1.8 & 0.6 & 0.0 &  4.5 \\
      $0.2<|\eta|<0.4$& 3& 2.5 & 0.6 & 0.0 &  4.4 \\
      $0.4<|\eta|<0.6$& 3& 2.7 & 0.3 & 0.0 &  4.4 \\
      $0.6<|\eta|<0.8$& 3& 2.5 & 0.3 & 0.0 &  4.4 \\
      $0.8<|\eta|<1.0$& 3& 1.9 & 0.6 & 0.1 &  4.4 \\
      $1.0<|\eta|<1.2$& 3& 2.4 & 1.0 & 0.1 &  4.9 \\
      $1.2<|\eta|<1.4$& 3& 2.6 & 0.8 & 0.1 &  5.4 \\
      $1.6<|\eta|<1.8$& 3& 3.1 & 0.8 & 0.1 &  9.2 \\
      $1.8<|\eta|<2.0$& 3& 2.0 & 1.6 & 0.2 &  8.7 \\
      $2.0<|\eta|<2.2$& 3& 2.0 & 2.6 & 0.3 & 10.0 \\
      $2.2<|\eta|<2.4$& 4& 2.9 & 2.4 & 0.3 & 12.5 \\
    \end{tabular}
    \caption{Summary of systematic errors, ($\times 10^{-3}$).  }
    \end{table}
  \end{center} 
  }

  \column{.4\textwidth}
    \begin{itemize}
      \item The efficiency ratio is the dominant source of systematic error on
      our measurement
      \item This is limited by statistics of the Z data sample
      \item We also give a full error correlation matrix with the measurement
      (backup)
    \end{itemize}
  \end{columns}
\end{frame}

\section{Results}

\begin{frame}{Results}
  \begin{center}
    \includegraphics[width=0.6\textwidth]{fig/powheg_results}
  \end{center}
\end{frame}

\begin{frame}{Conclusion}
  \begin{itemize}
    \item Lepton charge asymmetry has been measured in $W\to e\nu$ for $P_T>35$
    GeV in 11 $|\eta|$ bins with $840\ \mathrm{pb}^{-1}$ 
    \item $\sigma(Ae) = (6-14) x 10^{-3}$
    \item This measurement data with full error correlation matrix will be used by theorists to
    improve the knowledge of PDFs

  \end{itemize}
\end{frame}


\begin{frame}{Backup}
\end{frame}

\begin{frame}{Selection}
\tiny{
\begin{center}
    \begin{table}[htbp]
    \begin{tabular}{|lcc|} \hline
      \multicolumn{1}{|c}{Variable} & \multicolumn{1}{c}{cut value (barrel)}& \multicolumn{1}{c|}{cut value (endcap)}\\
        \hline   \hline
       \multicolumn{3}{|l|}{ID Cuts}\\ \hline
        H/E & 0.04 & 0.025 \\
        $\Delta\phi$ & 0.06 & 0.03 \\
        $\Delta\eta$ & 0.004 & 0.007  \\
        $\sigma_{\eta\eta}$ & 0.01 & 0.03 \\ \hline
      \multicolumn{3}{|l|}{Isolation Cuts}\\ \hline
       $ISO_{trk} / E_T $  & 0.09 & 0.04 \\
       $ISO_{ecal}/ E_T$  & 0.07 & 0.05 \\
       $ISO_{hcal}/ E_T$  & 0.10 & 0.025 \\ \hline
      \multicolumn{3}{|l|}{Conversion Rejection Cuts}\\ \hline
       Missing Hits  & \multicolumn{2}{c|}{$\leq 0$}\\
       Dist $||$ Dcot   & \multicolumn{2}{c|}{$>0.02$}\\
      \hline
    \end{tabular}
    \caption{\label{tab:elecuts} Electron selection.}% Electron selection variables and corresponding cut values.}
    \end{table}
\end{center}
    }
\end{frame}

\begin{frame}{Results}
\tiny{
\begin{center}
\begin{table}[tb]
     \caption{Summary of the measured charge asymmetry results.  All values are in units $\times 10^{-3}$.  }
       \begin{tabular}{l|l|cccc}
  & Measured & \multicolumn{4}{c} {Theory Prediction} \\
  & Asymmetry $(\mathcal{A})$ & CT10 & HERAPDF & MSTW &NNPDF \\ \hline
   $0.0<|\eta|<0.2$ &102$\pm$3$\pm$5 &$109^{+5}_{-5}$ &$106^{+4}_{-8}$ & $83^{+3}_{-5}$& 107$\pm$5\\
   $0.2<|\eta|<0.4$ &111$\pm$3$\pm$5 &$114^{+5}_{-5}$ &$110^{+4}_{-8}$ & $85^{+3}_{-5}$& 110$\pm$5\\
   $0.4<|\eta|<0.6$ &116$\pm$3$\pm$5 &$119^{+5}_{-5}$ &$115^{+4}_{-8}$ & $92^{+3}_{-5}$& 116$\pm$5\\
   $0.6<|\eta|<0.8$ &123$\pm$3$\pm$5 &$126^{+5}_{-5}$ &$122^{+4}_{-8}$ & $98^{+3}_{-5}$& 123$\pm$5\\
   $0.8<|\eta|<1.0$ &133$\pm$3$\pm$5 &$138^{+5}_{-6}$ &$132^{+4}_{-8}$ & $108^{+4}_{-5}$& 134$\pm$5\\
   $1.0<|\eta|<1.2$ &136$\pm$3$\pm$6 &$146^{+6}_{-6}$ &$140^{+5}_{-8}$ & $120^{+4}_{-5}$&145$\pm$5 \\
   $1.2<|\eta|<1.4$ &156$\pm$3$\pm$6 &$164^{+6}_{-7}$ &$153^{+5}_{-7}$ & $136^{+5}_{-5}$&158$\pm$5 \\
   $1.6<|\eta|<1.8$ &166$\pm$3$\pm$10 &$195^{+8}_{-9}$ &$181^{+5}_{-5}$ & $168^{+5}_{-5}$&190$\pm$4 \\
   $1.8<|\eta|<2.0$ &197$\pm$3$\pm$9 &$207^{+8}_{-10}$ &$196^{+4}_{-3}$ & $184^{+6}_{-5}$&206$\pm$4 \\
   $2.0<|\eta|<2.2$ &224$\pm$3$\pm$11 &$224^{+8}_{-11}$ &$211^{+5}_{-3}$ & $198^{+6}_{-5}$&219$\pm$4 \\
   $2.2<|\eta|<2.4$ &210$\pm$4$\pm$13 &$241^{+8}_{-12}$ &$225^{+9}_{-4}$ & $214^{+6}_{-5}$&231$\pm$5 \\
       \end{tabular}
   \end{table}
\end{center}
}
\end{frame}

%\tabcolsep 
\renewcommand\tabcolsep{0pt}
\renewcommand\arraystretch{1.6}
\begin{frame}{Covariance Matrix}
  \tiny{
  \begin{center}
    \begin{table}
    \begin{tabular}{l|ccccccccccc}
& \tiny{$\left[0.0,0.2\right]$}
& \tiny{$\left[0.2,0.4\right]$}
& \tiny{$\left[0.4,0.6\right]$} 
& \tiny{$\left[0.6,0.8\right]$}   
& \tiny{$\left[0.8,1.0\right]$}
& \tiny{$\left[1.0,1.2\right]$} 
& \tiny{$\left[1.2,1.4\right]$}
& \tiny{$\left[1.6,1.8\right]$}
& \tiny{$\left[1.8,2.0\right]$} 
& \tiny{$\left[2.0,2.2\right]$} 
& \tiny{$\left[2.2,2.4\right]$} \\ \hline
\tiny{$\left[0.0,0.2\right]$} & \alert{23.7} & 2.6 & 2.2 & 2.5 & 2.7 & 2.9 & 2.9 & 2.9 & 2.8 & 3.1 & 4.2 \\
\tiny{$\left[0.2,0.4\right]$} & 2.6 & \alert{26.2} & 2.6 & 2.9 & 3.1 & 3.3 & 3.4 & 3.2 & 3.0 & 3.9 & 4.7 \\
\tiny{$\left[0.4,0.6\right]$} & 2.2 & 2.6 & \alert{26.6} & 2.6 & 2.8 & 2.9 & 3.2 & 2.9 & 2.5 & 3.3 & 4.1 \\
\tiny{$\left[0.6,0.8\right]$} & 2.5 & 2.9 & 2.6 & \alert{25.6} & 3.3 & 3.4 & 3.7 & 3.3 & 2.8 & 3.7 & 4.7 \\
\tiny{$\left[0.8,1.0\right]$} & 2.7 & 3.1 & 2.8 & 3.3 & \alert{23.3} & 3.9 & 4.2 & 3.7 & 3.4 & 4.6 & 5.6 \\
\tiny{$\left[1.0,1.2\right]$} & 2.9 & 3.3 & 2.9 & 3.4 & 3.9 & \alert{30.8} & 4.5 & 4.1 & 4.0 & 5.7 & 6.8 \\
\tiny{$\left[1.2,1.4\right]$} & 2.9 & 3.4 & 3.2 & 3.7 & 4.2 & 4.5 & \alert{36.5} & 4.3 & 3.7 & 5.8 & 6.7 \\
\tiny{$\left[1.6,1.8\right]$} & 2.9 & 3.2 & 2.9 & 3.3 & 3.7 & 4.1 & 4.3 & \alert{94.9} & 3.8 & 5.1 & 6.2 \\
\tiny{$\left[1.8,2.0\right]$} & 2.8 & 3.0 & 2.5 & 2.8 & 3.4 & 4.0 & 3.7 & 3.8 & \alert{82.4} & 6.2 & 7.0 \\ 
\tiny{$\left[2.0,2.2\right]$} & 3.1 & 3.9 & 3.3 & 3.7 & 4.6 & 5.7 & 5.8 & 5.1 & 6.2 & \alert{110.7} & 10.3 \\
\tiny{$\left[2.2,2.4\right]$} & 4.2 & 4.7 & 4.1 & 4.7 & 5.6 & 6.8 & 6.7 & 6.2 & 7.0 & 10.3 & \alert{171.0} \\
    \end{tabular}
    \caption{Covariance matrix, ($\times 10^{-6}$).  }
    \end{table}
  \end{center}
  }
\end{frame}
\renewcommand\tabcolsep{6pt}
\renewcommand\arraystretch{1.}

\end{document}
