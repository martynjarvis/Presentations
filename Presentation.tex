\documentclass[t, 8pt]{beamer}
\usetheme{iclpt}
\usepackage{url}

\author{Martyn Jarvis}
\title[Electron Charge Asymmetry]{Electron Charge Asymmetry}

\begin{document}
\begin{frame}
  \titlepage
\end{frame}

\begin{frame}{Outline}
  \tableofcontents
\end{frame}

\section{Introduction}

\begin{frame}{Motivation}
  \begin{columns}[c]
  \column{.5\textwidth}
  \begin{itemize}
    \item In pp collisions, more W+ are expected than W- due to the excess of u valence quarks wrt d quarks.
    \item An asymmetry measurement as a function of boson rapidity can be used to constrain PDFs.
    \item LHC is exploring a new region in the x,Q2 plane
    \item The inclusive charge ratio measured by CMS to be $1.43 \pm 0.04$ in agreement with MSTW PDF predictions
  \end{itemize}
  \column{.5\textwidth}
    %\includegraphics[width=\textwidth]{fig/overtrain_old_same_script}
  \end{columns}
\end{frame}

\begin{frame}{Motivation}
  \begin{columns}[c]
  \column{.5\textwidth}
  \begin{itemize}
    \item Boson rapidity is not directly accessible
    \item Direct accessible measurement is the lepton charge asymmetry
    \begin{equation}
    A_{th}(\eta) = \frac{\frac{d\sigma}{d\eta}(W^{+}\to e^{+}\nu_{e}) - \frac{d\sigma}{d\eta}(W^{-}\to e^{-}\nu_{e})}
                        {\frac{d\sigma}{d\eta}(W^{+}\to e^{+}\nu_{e}) + \frac{d\sigma}{d\eta}(W^{-}\to e^{-}\nu_{e})}
    \end{equation}
    \item This asymmetry is given by the combination of the W production asymmetry and the well understood parity violation asymmetry in the W decay.
  \end{itemize}
  \column{.5\textwidth}
    %\includegraphics[width=\textwidth]{fig/overtrain_old_same_script}
  \end{columns}
\end{frame}

\section{History}

\begin{frame}{History}
  \begin{columns}[c]
  \column{.5\textwidth}
  \begin{itemize}
    \item 
  \end{itemize}
  \column{.5\textwidth}
    %\includegraphics[width=\textwidth]{fig/overtrain_old_same_script}
  \end{columns}
\end{frame}

\section{LHC Performance}

\begin{frame}{LHC Performance in 2011}
  \begin{columns}[c]
  \column{.5\textwidth}
  \begin{itemize}
    \item 
  \end{itemize}
  \column{.5\textwidth}
    %\includegraphics[width=\textwidth]{fig/overtrain_old_same_script}
  \end{columns}
\end{frame}

\section{CMS Detector}

\begin{frame}{CMS Detector}
    %\includegraphics[width=\textwidth]{fig/overtrain_old_same_script}
\end{frame}

\section{Analysis}

\begin{frame}{Analysis Overview}
  \begin{columns}[c]
  \column{.5\textwidth}
  \begin{itemize}
    \item $W \to e\nu$ characterised by 
    \begin{itemize}
      \item High Pt lepton 
      \item Missing transverse energy (MET) due to neutrino
    \end{itemize}
    \item Background contributions,
    \begin{itemize}
      \item EWK background ( $W \to e\nu$, Drell-Yan)
      \item $t\bar{t}$ background
      \item QCD background (multi-jet, photon+jet)
    \end{itemize}
    %\item Lepton pT> 35 GeV (limit set by single electron trigger) 
    \item The number of the electron / positron is extracted from a extended binned likelihood fit of the MET distribution
    \item 11 bin in $|\eta|$
    \begin{itemize}
      \item  $[0-0.2], [0.2-0.4], [0.4-0.6],
      [0.6-0.8], [0.8-1.0], [1.0-1.2], [1.2-1.4], [1.6-1.8], [1.8-2.0],
      [2.0-2.2], [2.2-2.4]$
      \item $[1.4-1.6]$ bin excluded because of the transition region between Ecal Barrel and Ecal Endcap
    \end{itemize}
  \end{itemize}
  \column{.5\textwidth}
    %\includegraphics[width=\textwidth]{fig/overtrain_old_same_script}
  \end{columns}
\end{frame}

\begin{frame}{Electron Selection}
  \begin{columns}[c]
  \column{.5\textwidth}
  \begin{itemize}
    \item Electron Selection:
    \begin{itemize}
      \item $P_T > 35 GeV$
    \end{itemize}
    \item Electron Identification
    \begin{itemize}
      \item Track cluster matching
      \item Shower shape and H/E
      \item Track, ECAL and HCAL isolation
      \item Conversion rejection
    \end{itemize}
    \item Z veto
    \begin{itemize}
      \item 2nd lepton with $P_T > 15$ GeV
    \end{itemize}
    \item Require that all three methods of charge assignment agree to reduce
    misassignment (Gaussian Sum Filter, Kalman Filter, Relative phi position of
    cluster center and first tracker hit)
  \end{itemize}
  \column{.5\textwidth}
    %table of cuts
    %table with 
    %\includegraphics[width=\textwidth]{fig/overtrain_old_same_script}
  \end{columns}
\end{frame}

\begin{frame}{Electron Results}
  \begin{columns}[c]
  \column{.5\textwidth}
  \begin{itemize}
    \item Apply electron selection, then extended binned maximum likelihood fit to the MET distribution for electrons and positrons separatly
    \item Static template shapes:
    \begin{itemize}
      \item Signal + EWK backgrounds : MC + correction from Z to ll recoil in data 
      \item QCD shape is determined using a signal-free control sample obtained by inverting a subset of the electron id criteria
    \end{itemize}
  \end{itemize}
  \column{.5\textwidth}
    % FIT here
    %\includegraphics[width=\textwidth]{fig/overtrain_old_same_script}
  \end{columns}
\end{frame}

\section{Systematics}

\begin{frame}{Systematics and Corrections}
  \begin{columns}[c]
  \column{.5\textwidth}
  \begin{itemize}
    \item Four main sources of systematic uncertainty in both channels:
    \begin{itemize}
      \item Signal estimation method
      \item Rate of lepton charge misassignment
      \item Relative efficiencies between l+ and l−
      \item Lepton momentum (energy) scale and resolution
    \end{itemize}
  \end{itemize}
  \column{.5\textwidth}
    %\includegraphics[width=\textwidth]{fig/overtrain_old_same_script}
  \end{columns}
\end{frame}

\begin{frame}{Signal Extraction Method}
  \begin{columns}[c]
  \column{.5\textwidth}
  \begin{itemize}
    \item Systematic uncertainties evaluated by varying the shapes used in the fits
    \item Measured the effect of changing EWK and top backgrounds
  \end{itemize}
  \column{.5\textwidth}
    %\includegraphics[width=\textwidth]{fig/overtrain_old_same_script}
  \end{columns}
\end{frame}

\begin{frame}{Corrections}
  \begin{itemize}
    \item To compare with the theoretical value, the measured lepton asymmetry has to be corrected for experimental effects:
    \begin{itemize}
      \item Charge mis-id $(\omega)$: 
      \begin{itemize}
        \item Introduces a dilution factor to the real asymmetry
        \item Evaluated from the ratio same-sign / opposite-sign Z yield
        %\item  $R_{ij}$=fraction of the same sign events with one electron in the η bin i and the other in the bin j
        %\item From a simultaneous fit to the 66 values of $R_{ij}$, the value of $\omega$ is evaluated for each $\eta$ bin
        \item Statistical error propogated to asymmetry as systematic error
      \end{itemize}
      \item Relative detection efficiency $(R = \epsilon^+/\epsilon^-)$ between electrons and positrons: 
      \begin{itemize}
        \item the detection efficiency is different for electrons and positrons and produces a bias in the measured asymmetry
        \item Measured with tag and probe.
        \item Systematic error (energy scale and signal shape) cancel out in the ratio R, only the statistical error is propagated to R
      \end{itemize}
      \begin{equation}
        \mathcal{A}_R=
          \frac{1}{1-2\omega}\frac{ \mathcal{A}_M\left(R+1\right) - \left(R-1\right)}{\left(R+1\right)-\mathcal{A}_M \left(R-1\right)}
          \simeq 
          \frac{1}{1-2\omega}\left(\mathcal{A}_M -\frac{\left(R-1\right)\left(1-\mathcal{A}_M^2\right)}{2}\right)
      \end{equation}
    \end{itemize}
  \end{itemize}
\end{frame}

\begin{frame}{Energy Scale and Resolution}
    \begin{itemize}
      \item The electron energy resolution and scale can introduce a bias on the asymmetry, due to the effect of resolution for leptons with a transverse momentum close to Pt cut.
      \item The correction factor has been evaluated by comparing the asymmetry at gen level and the asymmetry after simulating the particle-matter interaction
      \begin{equation}
      P_T^{sim} = P_T^{gen} \otimes Res^{MC} \otimes Res^{Data/MC}
      \end{equation}
      \item where:
      \begin{itemize} 
        \item $P_T^{gen}$ is from resbos with CT10
        \item $Res^{MC}$ is from a $(W\to e\nu)$ MC sample
        \item $Res^{Data/MC}$ is from a residual data/MC correction 
      \end{itemize}
    \end{itemize}
\end{frame}

\begin{frame}{Summary of Systematic Uncertainties}
    %\includegraphics[width=\textwidth]{fig/overtrain_old_same_script}
\end{frame}

\begin{frame}{Results}
  \begin{columns}[c]
  \column{.5\textwidth}
  \column{.5\textwidth}
    %\includegraphics[width=\textwidth]{fig/overtrain_old_same_script}
  \end{columns}
\end{frame}

\begin{frame}{Conclusion}
  \begin{columns}[c]
  \column{.5\textwidth}
  \column{.5\textwidth}
    %\includegraphics[width=\textwidth]{fig/overtrain_old_same_script}
  \end{columns}
\end{frame}


\begin{frame}{Backup}
\end{frame}

\end{document}
