\documentclass[t, 8pt]{beamer}
\usetheme{iclpt}
\usepackage{url}

\author{Martyn Jarvis}
\title[Input Variables]{How to Select Input Variables}

\begin{document}
% \begin{frame}
%   \titlepage
% \end{frame}

\begin{frame}{Outline}
  \tableofcontents
\end{frame}

\section{Removing Mass from the BDT}
\begin{frame}{Removing Mass from the BDT}
\end{frame}

\begin{frame}{Removing Mass from the BDT}
  \begin{itemize}  
  \item Pull back a little from the fully aggressive approach and instead focus on the background modelling aspect.
  \item Used a likelihod method with two inputs:
  \begin{itemize}  
    \item Output of a BDT trained with out the mass as an input. Here the MIT
    bdt weights are used.
    \item $\Delta M/M$ variable
  \end{itemize}
  \item Produces a single variable and allows for our background modeling approach.
  \end{itemize}
\end{frame}

\begin{frame}{Likelihood Inputs}
\begin{center}
    % Likelihood inputs 
    \includegraphics[width=\textwidth]{fig/likelihood_input}
\end{center}
\end{frame}

\begin{frame}{Likelihood Outputs}
\begin{center}
    % Likelihood Outputs
    \includegraphics[width=0.75\textwidth]{fig/likelihood_output}
\end{center}
\end{frame}

\begin{frame}{Limit Plot}
\begin{center}
    % Limit Plot
    \includegraphics[width=0.75\textwidth]{fig/limit_likelihood}
\end{center}
\end{frame}

\begin{frame}{Conclusions}

  \begin{itemize}  
  \item Correlations between Delta M Over M and BDT output
  \item Might not be a huge effect with a $2 \%$ mass window
  \item Could be taken in to account using a second BDT instead of a likelihood
  method
  \end{itemize}
\end{frame}

\section{Thoughts about Input Variables}

\begin{frame}{Thoughts about Input Variables}
  \begin{itemize}  
    \item The variables we use can be split in to 3 categories:
    \begin{itemize}  
      \item variables that contain information about the production mechanism
      \item variables that contain information about the decay 
      \item variables that contain information about the reconstruction of the
      event
    \end{itemize}
    \item Some variables combine information from different categories ($\delta
    \phi$)
  \end{itemize}
\end{frame}

\begin{frame}{Production Mechanism}
  \begin{columns}[c]
  \column{.5\textwidth}
  \begin{itemize}  
    \item There are three main production mechanisms for Higgs at high energy
    proton proton colliders:
    \begin{itemize}  
      \item Gluon Fusion
      \item Vector Boson Fusion
      \item Asociative Production
    \end{itemize}
    \item The variables that contain information about the production mechanism will be
    the variables that fully describe the information about the reconstructed Higgs,
    the Eta and the PT.  
    \item Studied these two variables for each production mechanism
    \item Signal MC samples are reweighted to remove the dependence on eta when
    studying PT and vice versa.
    \item Signal MC has been reweighted to remove dependence on Mass
  \end{itemize}
  \column{.5\textwidth}
    \includegraphics[width=\textwidth]{fig/production}
  \end{columns}
\end{frame}

\begin{frame}{Gluon Fusion}
  \begin{itemize}  
    \item Expect Higgs to be more central than background. 
  \end{itemize}
    % higgs eta and higgs pt
    \begin{center}
    \includegraphics[width=0.45\textwidth]{fig/gluglu_eta}
    \includegraphics[width=0.45\textwidth]{fig/gluglu_pt}
    \end{center}
\end{frame}

\begin{frame}{Vector Boson Fusion}
  \begin{itemize}  
    \item Again Expect Higgs to be more central than background. 
    \item Expect Higgs to be produced with higher PT due to recoil from the
    boson in the production.
  \end{itemize}
    \begin{center}
    \includegraphics[width=0.45\textwidth]{fig/vbf_eta}
    \includegraphics[width=0.45\textwidth]{fig/vbf_pt}
    \end{center}
\end{frame}

\begin{frame}{Decay Mechanism}
  \begin{columns}[c]
  \column{.5\textwidth}
  \begin{itemize}  
    \item Signal MC has been reweighted to remove dependence on Mass
    \item $\cos \theta \ast $
    \begin{itemize}  
      \item Defined as 
      \begin{equation}
      \cos \theta \ast = | E^{lead}-E^{sublead} | / P^{Higgs}
      \end{equation}
    \end{itemize}  
  \end{itemize}
  \column{.5\textwidth}
    \includegraphics[width=\textwidth]{fig/costhetastar}
  \end{columns}
\end{frame}

\begin{frame}{Event Reconstruction Variables}
 \begin{columns}[c]
 \column{.5\textwidth}
 \begin{itemize}  
   \item Variables that categorise the events
   \item 3 new variables studied
   \begin{itemize}  
     \item $\sigma M_{Correct Vertex} / M$
     \item $\sigma M_{Wrong Vertex} / M$
     \item $P_{vertex} =  1.-0.49*(evtmva+1.0)$
   \end{itemize}
   \item Removed variables:
   \begin{itemize}  
     \item $\sigma M / M$ 
     \item $Max(\eta_1,\eta_2)$
     \item $Min(r9_1,r9_2)$
   \end{itemize}
 \end{itemize}
 \column{.5\textwidth}
    % Limit Plot
    % p vertex input variable plot
    \includegraphics[width=\textwidth]{fig/vtx_prob}
  \end{columns}
\end{frame}

\begin{frame}{$\sigma M / M$}
  \begin{columns}[c]
  \column{.5\textwidth}
    \includegraphics[width=\textwidth]{fig/sigmaMcorrect}
  \column{.5\textwidth}
    \includegraphics[width=\textwidth]{fig/sigmaMwrong}
  \end{columns}
\end{frame}

\begin{frame}{Conclusions}

  \begin{itemize}  
  \item Justified in including the Higgs PT and eta in to the training.
  \item Cos Theta Star offers good separation over.
  \item Others need to be studied in more detail.
  \end{itemize}
\end{frame}

\begin{frame}{Backup} 


\end{frame}


\begin{frame}{Limit with New Variables} 
\begin{center}
    % p vertex input variable plot
    \includegraphics[width=\textwidth]{fig/limit_sm_Asymptotic_ratio_var3}
\end{center}
\end{frame}

\begin{frame}{Limit with Likelihood} 
\begin{center}
    % p vertex input variable plot
    \includegraphics[width=\textwidth]{fig/limit_sm_Asymptotic_ratio_likelihood}
\end{center}
\end{frame}

\begin{frame}{Likelihood Correlations}
  \begin{columns}[c]
  \column{.5\textwidth}
    \includegraphics[width=\textwidth]{fig/CorrelationMatrixS_123_all}
  \column{.5\textwidth}
    \includegraphics[width=\textwidth]{fig/CorrelationMatrixB_123_all}
  \end{columns}
\end{frame}


\begin{frame}{BDT output}
\begin{center}
    % p vertex input variable plot
    \includegraphics[width=0.75\textwidth]{fig/overtrain_3var}
\end{center}
\end{frame}

\begin{frame}{Limit Plot}
\begin{center}
    % Limit Plot
    \includegraphics[width=0.75\textwidth]{fig/limit_var3}
\end{center}
\end{frame}

\end{document}
